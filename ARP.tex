\documentclass{article}
\usepackage[utf8]{inputenc}
\usepackage{amsmath}
\usepackage{amsfonts}
\usepackage{graphicx} 
\graphicspath{ {./Images/} } %tells the program what folder to find the images in

\title{Proportional Approval Voting\\DRAFT}
\author{
  Lim Devon\\
  \texttt{h2010076@nushigh.edu.sg}
  \and
  Lee Hong Hao Casper\\
  \texttt{h2010069@nushigh.edu.sg}
  \and
  Kang Jun Jie Joseph\\
  \texttt{h2010059@nushigh.edu.sg}
}
\date{\today}

\begin{document}

\maketitle

\section{Introduction}
The purpose of this document is to test out some of the features of \LaTeX{}. Compile this document (Keyboard Shortcut Ctrl + F9), read the notes and compare the code to the final product to learn how to use these commands.

\section{Equations}
This section will be testing some different ways to write equations in different ways. 

\subsection{Simple Equations}
Here is an example of an equation
%using the command \begin{equation} always numbers the equation, label just allows you to reference later%
\begin{equation} \label{luna}
P(x)=ax^2+bx+c
\end{equation}

The equation \ref{luna} is an example of a quadratic equation.

\subsection{Splitting an equation}

%This is how you can add a comment without it showing in your final document.  The \\ starts a new line. The & symbol tells the program where you want your two equations to align. split allows you to label the group of equations%
\begin{equation}\label{eq2}
\begin{split}
V&=\frac{4}{3}\pi r^3 \\
&= \frac{4\pi r^3}{3}
\end{split}
\end{equation}

Equation \ref{eq2} shows that you can write the equation for the volume of a sphere in multiple ways.

\subsection{Aligning}

Here is an example of aligning the equations with the = signs.
\begin{align*} 
2x+3&=3\\
3x+2&=2x+3
\end{align*}\\

Below is another example but with multiple equations.
\begin{align*}
2x&=3y  &   A&=B    &   3g&=h+4 \\
x+2&=y  &   A&=2B+1 &   g+1&=2h
\end{align*}\\

Here is the last example of this section where the equations are centered but not aligned.
\begin{gather*} 
5x+3=4y-3 \\
2x=y+1 
\end{gather*}\\


\subsection{Other ways of writing math}
%There are some more basic commands to write math as well such as \( \), \[ \], $ $, or \begin{math} \end{math}
You can also write equations such as \(A=\pi r^2\) in line like this or an equation such as $A=l*w$ in line using the other command that you can see in the code.\\\\

To write equations not in line you can also write it like this: \[A=\frac{1}{2}h(b_1+b_2)\]\\\\\\

\section{Questions}

Let there be an election $\boldsymbol{E} = (\boldsymbol{C}, \boldsymbol{V}, \boldsymbol{A})$.
$\boldsymbol{C}$ is the set of all candidates.
$\boldsymbol{V}$ is the set of all voters.
$\boldsymbol{A}$ is the set of all of the voters' Approval Value Functions.\\

$App^{\boldsymbol{E}}_{v}(c)$ is an Approval Value Function for voter $v \in \boldsymbol{V}$ in election $\boldsymbol{E}$.
\begin{equation}
App^{\boldsymbol{E}}_{v} : \boldsymbol{C} \mapsto [-1,1]
\end{equation}
Where a negative value means the voter believes the candidate will have a bad impact, a positive value means the voter believes the candidate will have a positive impact, and 0 means the voter is apathetic towards the candidate. -1 is the worst possible impact and +1 is the best possible impact a candidate may have. It is not neccessary that the least liked and most liked candidates are placed at -1 and +1 respectively, they may just be assigned an arbitrary value between -1 and 1.\\

$Res^{\boldsymbol{E}}_{S}(c)$ is a function, that given election $\boldsymbol{E}$ and voting strategy $S$, provides the result of the vote for candidate $c \in \boldsymbol{C}$.
\begin{gather}
Res^{\boldsymbol{E}}_{S} : \boldsymbol{C} \mapsto [0,1]\\
\sum^{}_{c \in \boldsymbol{C}}{Res^{\boldsymbol{E}}_{S}(c)} = 1
\end{gather}

${IRV}^{\boldsymbol{E}}_{S}(v)$ is a function, that given election $\boldsymbol{E}$ and voting strategy $S$, provides a numerical utility or satisfaction value with the results for voter $v \in \boldsymbol{V}$. We call the output of this function the Individual Reward Value.
\begin{equation}
{IRV}^{\boldsymbol{E}}_{S}(v) = \sum^{}_{c \in \boldsymbol{C}}{(Res^{\boldsymbol{E}}_{S}(c) \cdot App^{\boldsymbol{E}}_{v}(c))}
\end{equation}
It is trivial to show that ${IRV}^{\boldsymbol{E}}_{S} : \boldsymbol{V} \mapsto [-1,1]$. The proof exists but I'm gonna add it later.


\subsection{Question 1}

Let there be a Socially Optimal outcome. It will be defined as the case where
\begin{gather}
App^{\boldsymbol{E}}_{pop}(c) = \sum^{}_{v \in \boldsymbol{V}}{App^{\boldsymbol{E}}_{v}(c)}\\
c_{w} = {argmax}_{c \in \boldsymbol{C}} App^{\boldsymbol{E}}_{pop}(c)\\
Res^{\boldsymbol{E}}_{S}(c_{w}) = 1
\end{gather}\\

Let there be a Social-Positive Optimal outcome. It will be defined as the case where the value of the function below is maximised.
\begin{equation}
\sum^{}_{v \in \boldsymbol{V}}{\left(
\begin{cases}
1, & \sum^{}_{c \in \boldsymbol{C}}{(Res^{\boldsymbol{E}}_{S}(c) \cdot App^{\boldsymbol{E}}_{v}(c))} > 0\\
0, & {otherwise}
\end{cases}
\right)}
\end{equation}\\

Let us define the Population Reward Value as
\begin{equation}
\sum^{}_{v \in \boldsymbol{V}}{\left(
\sum^{}_{c \in \boldsymbol{C}}{\left(
Res^{\boldsymbol{E}}_{S}(c) \cdot App^{\boldsymbol{E}}_{v}(c)
\right)}
\right)}
\end{equation}\\

Are there bounds (either upper or lower) on the difference between the population reward values of the Socially Optimal and Social-Positive Optimal outcomes?

It is known that the Socially Optimal outcome maximises PRV. Wei Shen Me? Yin Wei
\begin{equation}
\begin{aligned}
PRV
&= \sum^{}_{v \in \boldsymbol{V}}{\left(\sum^{}_{c \in \boldsymbol{C}}{\left(Res^{\boldsymbol{E}}_{S}(c) \cdot App^{\boldsymbol{E}}_{v}(c)\right)}\right)}\\
&= \sum^{}_{c \in \boldsymbol{C}}{\left(\sum^{}_{v \in \boldsymbol{V}}{\left(Res^{\boldsymbol{E}}_{S}(c) \cdot App^{\boldsymbol{E}}_{v}(c)\right)}\right)}\\
&= \sum^{}_{c \in \boldsymbol{C}}{\left(Res^{\boldsymbol{E}}_{S}(c) \cdot \sum^{}_{v \in \boldsymbol{V}}{\left(App^{\boldsymbol{E}}_{v}(c)\right)}\right)}
\end{aligned}
\end{equation}


\subsection{Strategies and additional outcomes}

\subsubsection{Outcomes}

Let there be a Social-Sign Optimal outcome. It will be defined as the case where the value of the function below is maximised.
\begin{equation}
\sum^{}_{v \in \boldsymbol{V}}{\left(
\begin{cases}
1, & \sum^{}_{c \in \boldsymbol{C}}{(Res^{\boldsymbol{E}}_{S}(c) \cdot App^{\boldsymbol{E}}_{v}(c))} > 0\\
-1, & \sum^{}_{c \in \boldsymbol{C}}{(Res^{\boldsymbol{E}}_{S}(c) \cdot App^{\boldsymbol{E}}_{v}(c))} < 0\\
0, & {otherwise}
\end{cases}
\right)}
\end{equation}\\

Let there be a Social-Negative Optimal outcome. It will be defined as the case where the value of the function below is maximised.
\begin{equation}
\sum^{}_{v \in \boldsymbol{V}}{\left(
\begin{cases}
-1, & \sum^{}_{c \in \boldsymbol{C}}{(Res^{\boldsymbol{E}}_{S}(c) \cdot App^{\boldsymbol{E}}_{v}(c))} < 0\\
0, & {otherwise}
\end{cases}
\right)}
\end{equation}\\

\subsubsection{Strategies}

Strategies differ from outcomes in that strategies lead to outcomes. Therefore, our Optimal Outcomes are theoretical situations that may in fact have no simple strategy that leads to it. It would therefore be of great interest to find the social costs between the outcomes of simple strategies and optimal outcomes.\\

\noindent Here is a list of a few simple strategies we will analyse:
\begin{itemize}
  \item A voter only votes for their highest ranked options. (Strategy 1)
  \item A voter only votes for all candidates except their lowest ranked option. (Strategy 2)
  \item A voter only votes for the options with a positive individual reward weight. (Strategy 3)
\end{itemize}

\subsubsection{Strategy 3}

Let there be an election $\boldsymbol{E}$ with a set of candidates $\boldsymbol{C}$ and a set of voters $\boldsymbol{V}$. 
Let $\boldsymbol{C}$ be split into subsets $\boldsymbol{C}_{1}$, $\boldsymbol{C}_{2}$ and $\boldsymbol{C}_{3}$ (i.e. $\boldsymbol{C}_{1}, \boldsymbol{C}_{2}, \boldsymbol{C}_{3} \subset \boldsymbol{C}$ and $\boldsymbol{C}_{1}+\boldsymbol{C}_{2}+\boldsymbol{C}_{3} \equiv \boldsymbol{C}$) and let $\boldsymbol{V}$ be split into subsets $\boldsymbol{V}_{1}$, $\boldsymbol{V}_{2}$, and $\boldsymbol{V}_{3}$ (i.e. $\boldsymbol{V}_{1}, \boldsymbol{V}_{2}, \boldsymbol{V}_{3} \subset \boldsymbol{V}$ and $\boldsymbol{V}_{1}+\boldsymbol{V}_{2}+\boldsymbol{V}_{3} \equiv \boldsymbol{V}$) with sizes ${m}_{1}$, ${m}_{2}$ and ${m}_{3}$ respectively.
Let the subsets fulfil the following conditions:\\

\begin{itemize}
    \item For all $v \in \boldsymbol{V}_{1}$:\\
    \begin{gather}
    \begin{split}
        \sum^{}_{c \in \boldsymbol{C}_{1}}{App^{\boldsymbol{E}}_{v}(c)} \leq -\frac{{m}_{2}}{{m}_{1}}\sum^{}_{c \in \boldsymbol{C}_{2}}{App^{\boldsymbol{E}}_{v}(c)}
    \end{split}\\
    App^{\boldsymbol{E}}_{v}(c) \geq 0, c \in \boldsymbol{C}_{1}\\
    App^{\boldsymbol{E}}_{v}(c) \leq 0, c \in \boldsymbol{C}_{2} \cup \boldsymbol{C}_{3}
    \end{gather}
    \item For all $v \in \boldsymbol{V}_{2}$:\\
    \begin{gather}
    \begin{split}
        \sum^{}_{c \in \boldsymbol{C}_{2}}{App^{\boldsymbol{E}}_{v}(c)} \leq -\frac{{m}_{1}}{{m}_{2}}\sum^{}_{c \in \boldsymbol{C}_{1}}{App^{\boldsymbol{E}}_{v}(c)}
    \end{split}\\
    App^{\boldsymbol{E}}_{v}(c) \geq 0, c \in \boldsymbol{C}_{2}\\
    App^{\boldsymbol{E}}_{v}(c) \leq 0, c \in \boldsymbol{C}_{1} \cup \boldsymbol{C}_{3}
    \end{gather}
    \item For all $v \in \boldsymbol{V}_{3}$:\\
    \begin{gather}
        App^{\boldsymbol{E}}_{v}(c) \leq 0, c \in \boldsymbol{C}
    \end{gather}
\end{itemize}

Given these conditions and using strategy 3,
\begin{equation}
    {IRV}^{\boldsymbol{E}}_{S}(v) = \sum^{}_{c \in \boldsymbol{C}}{(Res^{\boldsymbol{E}}_{S}(c) \cdot App^{\boldsymbol{E}}_{v}(c))}\\
    = \sum^{}_{c \in \boldsymbol{C}_{1}}{(Res^{\boldsymbol{E}}_{S}(c) \cdot App^{\boldsymbol{E}}_{v}(c))}+\sum^{}_{c \in \boldsymbol{C}_{2}}{(Res^{\boldsymbol{E}}_{S}(c) \cdot App^{\boldsymbol{E}}_{v}(c))}+\sum^{}_{c \in \boldsymbol{C}_{3}}{(Res^{\boldsymbol{E}}_{S}(c) \cdot App^{\boldsymbol{E}}_{v}(c))}
\end{equation}\\

As $App^{\boldsymbol{E}}_{v}(c) \leq 0$ for all $c \in \boldsymbol{C}_{3}$, $\sum^{}_{c \in \boldsymbol{C}_{3}}{(Res^{\boldsymbol{E}}_{S}(c) \cdot App^{\boldsymbol{E}}_{v}(c))}=0$. 
Additionally, as $App^{\boldsymbol{E}}_{v}(c) \leq 0$ for all $c$ when $v \in \boldsymbol{V}_{3}$, ${IRV}^{\boldsymbol{E}}_{S}(v) \leq 0$ for all $v \in \boldsymbol{V}_{3}$. 
The total number of votes $x$ will be equal to ${m}_{1}{c}_{1}+{m}_{2}{c}_{2}$, where ${c}_{1}$ and ${c}_{2}$ are the sizes of $\boldsymbol{C}_{1}$ and $\boldsymbol{C}_{2}$ respectively. This is because given strategy 3 all $v \in \boldsymbol{V}_{1}$ or $\boldsymbol{V}_{2}$ will vote for all $c \in \boldsymbol{C}_{1}$ or $\boldsymbol{C}_{2}$ respectively i.e. ${c}_{1}$ candidates receive ${m}_{1}$ votes each and ${c}_{2}$ candidates receive ${m}_{2}$ votes each.\\
For $v \in \boldsymbol{V}_{1}$:
\begin{equation}
\begin{split}
    {IRV}^{\boldsymbol{E}}_{S}(v) = \sum^{}_{c \in \boldsymbol{C}}{(Res^{\boldsymbol{E}}_{S}(c) \cdot App^{\boldsymbol{E}}_{v}(c))}\\
    = \sum^{}_{c \in \boldsymbol{C}_{1}}{(Res^{\boldsymbol{E}}_{S}(c) \cdot App^{\boldsymbol{E}}_{v}(c))}+\sum^{}_{c \in \boldsymbol{C}_{2}}{(Res^{\boldsymbol{E}}_{S}(c) \cdot App^{\boldsymbol{E}}_{v}(c))}\\
    = \sum^{}_{c \in \boldsymbol{C}_{1}}{(\frac{{m}_{1}}{x} \cdot App^{\boldsymbol{E}}_{v}(c))}+\sum^{}_{c \in \boldsymbol{C}_{2}}{(\frac{{m}_{2}}{x} \cdot App^{\boldsymbol{E}}_{v}(c))}\\
    = \frac{{m}_{1}}{x} \cdot \sum^{}_{c \in \boldsymbol{C}_{1}}{App^{\boldsymbol{E}}_{v}(c)}+\frac{{m}_{2}}{x} \cdot \sum^{}_{c \in \boldsymbol{C}_{2}}{App^{\boldsymbol{E}}_{v}(c)}\\
    \leq \frac{{m}_{1}}{x} \cdot (-\frac{{m}_{2}}{{m}_{1}}\sum^{}_{c \in \boldsymbol{C}_{2}}{App^{\boldsymbol{E}}_{v}(c)})+\frac{{m}_{2}}{x} \cdot \sum^{}_{c \in \boldsymbol{C}_{2}}{App^{\boldsymbol{E}}_{v}(c)}\\
    = -\frac{{m}_{2}}{x} \cdot \sum^{}_{c \in \boldsymbol{C}_{2}}{App^{\boldsymbol{E}}_{v}(c)}+\frac{{m}_{2}}{x} \cdot \sum^{}_{c \in \boldsymbol{C}_{2}}{App^{\boldsymbol{E}}_{v}(c)}\\
    = 0
\end{split}     
\end{equation}\\

A similar proof exists for $v \in \boldsymbol{V}_{1}$.\\
Hence ${IRV}^{\boldsymbol{E}}_{S}(v) \leq 0$ for all $v$, showing that choosing strategy 3 with these conditions will lead to the least Social-Positive, Social-Sign and Social-Sign optimal solutions. 

\subsubsection{More Questions!}

Are there bounds between the social cost of these strategies and outcomes compared to the utilitarian Socially Optimal outcome?\\
Are there bounds between the difference of social costs among these outcomes?

\subsection{Question 2}
We can show that a voter voting for the highest weighted option will never be detrimental to their Individual Reward Value.\\

Consider a voter $v_{1}$. Let $c_{1}$ be their highest weighted candidate, ${argmax}_{c \in \boldsymbol{C}}App^{\boldsymbol{E}}_{v_{i}}(c)$. Now consider 2 cases: (1) $v_{1}$ approves of $c_{1}$, (2) $v_{1}$ does not approve of $c_{1}$. Let $a_{c}$ be the number of approvals candidate $c \in \boldsymbol{C}$ has excluding $v_{1}$'s vote for $c_{1}$. Let $m$ be the total number of votes excluding $v_{1}$'s vote for $c_{1}$.\\

In case (1), 
\begin{equation}
\begin{aligned}
{IRV}^{\boldsymbol{E}}_{(1)}(v_1) 
&= {Res}^{\boldsymbol{E}}_{S}(c_1)\cdot{App}^{\boldsymbol{E}}_{v_1}(c_1) + \sum^{|\boldsymbol{C}|}_{n = 2}{\left(\frac{a_{c_n}}{m(m+1)}\cdot{App}^{\boldsymbol{E}}_{v_1}(c_{n})\right)}\\
&= \frac{a_1+1}{m+1}\cdot{App}^{\boldsymbol{E}}_{v_1}(c_1) + \sum^{|\boldsymbol{C}|}_{n = 2}{\left(\frac{a_{c_n}}{m(m+1)}\cdot{App}^{\boldsymbol{E}}_{v_1}(c_{n})\right)}
\end{aligned}  
\end{equation}

In case (2), 
\begin{equation}
\begin{aligned}
{IRV}^{\boldsymbol{E}}_{(1)}(v_1) 
&= {Res}^{\boldsymbol{E}}_{S}(c_1)\cdot{App}^{\boldsymbol{E}}_{v_1}(c_1) + \sum^{|\boldsymbol{C}|}_{n = 2}{\left(\frac{a_{c_n}}{m(m+1)}\cdot{App}^{\boldsymbol{E}}_{v_1}(c_{n})\right)}\\
&= \frac{a_1}{m}\cdot{App}^{\boldsymbol{E}}_{v_1}(c_1) + \sum^{|\boldsymbol{C}|}_{n = 2}{\left(\frac{a_{c_n}}{m(m+1)}\cdot{App}^{\boldsymbol{E}}_{v_1}(c_{n})\right)}
\end{aligned}  
\end{equation}

Therefore,
\begin{equation}\label{IRV_DIFF}
\begin{aligned}
{IRV}^{\boldsymbol{E}}_{(2)}(v_1) - {IRV}^{\boldsymbol{E}}_{(1)}(v_1)
&=  \sum^{|\boldsymbol{C}|}_{n = 2}{\left(\frac{a_{c_{n}}}{m(m+1)}\cdot{App}^{\boldsymbol{E}}_{v_1}(c_{n})\right)}\\
&+ \left(\frac{a_1}{m}-\frac{a_{1}+1}{m+1}\right)\cdot{App}^{\boldsymbol{E}}_{v_1}(c_1)\\
\end{aligned}  
\end{equation}

Case (2) will only yield a larger Individual Reward Value than Case (1) if 
\begin{gather}
\sum^{|\boldsymbol{C}|}_{n = 2}{\left(\frac{a_{c_{n}}}{m(m+1)}\cdot{App}^{\boldsymbol{E}}_{v_1}(c_{n})\right)} + \left(\frac{a_1}{m}-\frac{a_{1}+1}{m+1}\right)\cdot{App}^{\boldsymbol{E}}_{v_1}(c_1) > 0\nonumber\\
\sum^{|\boldsymbol{C}|}_{n = 2}{\left(\frac{a_{c_{n}}}{m(m+1)}\cdot{App}^{\boldsymbol{E}}_{v_1}(c_{n})\right)} > \left(\frac{a_1}{m}-\frac{a_{1}+1}{m+1}\right)\cdot{App}^{\boldsymbol{E}}_{v_1}(c_1)
\end{gather}

The maximum value of $\sum^{|\boldsymbol{C}|}_{n = 2}{\left(\frac{a_{c_{n}}}{m(m+1)}\cdot{App}^{\boldsymbol{E}}_{v_1}(c_{n})\right)}$ is attained when ${App}^{\boldsymbol{E}}_{v_1}(c_{n}) = 1 \forall {c} \in \boldsymbol{C} \setminus \{c_1\}$. So,
\begin{equation}
\sum^{|\boldsymbol{C}|}_{n = 2}{\left(\frac{a_{c_{n}}}{m(m+1)}\cdot{App}^{\boldsymbol{E}}_{v_1}(c_{n})\right)} = \sum^{|\boldsymbol{C}|}_{n = 2}{\left(\frac{a_{c_{n}}}{m(m+1)}\right)} = \frac{\sum^{|\boldsymbol{C}|}_{n = 2}{a_{c_{n}}}}{m(m+1)} = \frac{m-a_1}{m(m+1)}
\end{equation}

Therefore,
\begin{equation}
\sum^{|\boldsymbol{C}|}_{n = 2}{\left(\frac{a_{c_{n}}}{m(m+1)}\cdot{App}^{\boldsymbol{E}}_{v_1}(c_{n})\right)} \leq \beta\cdot\sum^{|\boldsymbol{C}|}_{n = 2}{\left(\frac{a_{c_{n}}}{m(m+1)}\right)} = \beta\cdot\frac{m-a_1}{m(m+1)}
\end{equation}
Where $\beta$ is ${max}_{c \in \boldsymbol{C} \setminus \{c_1\}}\left({App}^{\boldsymbol{E}}_{v_1}(c)\right)$. Note that since we have defined $c_1$ to be the highest weighted candidate for $v_1$, $\beta \leq {App}^{\boldsymbol{E}}_{v_1}(c_1)$.\\

Therefore, substituting this inequality into \eqref{IRV_DIFF},
\begin{equation}
\begin{aligned}
{IRV}^{\boldsymbol{E}}_{(2)}(v_1) - {IRV}^{\boldsymbol{E}}_{(1)}(v_1)
&=  \sum^{|\boldsymbol{C}|}_{n = 2}{\left(\frac{a_{c_{n}}}{m(m+1)}\cdot{App}^{\boldsymbol{E}}_{v_1}(c_{n})\right)}\\
&+ \left(\frac{a_1}{m}-\frac{a_{1}+1}{m+1}\right)\cdot{App}^{\boldsymbol{E}}_{v_1}(c_1)\\
&\leq \beta\cdot\frac{m-a_1}{m(m+1)} + \left(\frac{a_1}{m}-\frac{a_{1}+1}{m+1}\right)\cdot{App}^{\boldsymbol{E}}_{v_1}(c_1)\\
&= \beta\cdot\frac{m-a_1}{m(m+1)} - \frac{m-a_1}{m(m+1)}\cdot{App}^{\boldsymbol{E}}_{v_1}(c_1)\\
&= \left(\beta - {App}^{\boldsymbol{E}}_{v_1}(c_1)\right)\left(\frac{m-a_1}{m(m+1)}\right)
\end{aligned} 
\end{equation}\\

Since $\beta \leq {App}^{\boldsymbol{E}}_{v_1}(c_1)$,
\begin{equation}
     \beta - {App}^{\boldsymbol{E}}_{v_1}(c_1) \leq 0
\end{equation}
and since $m \geq a_1$ and $m \in \mathbb{Z}^+$,
\begin{equation}
     \frac{m-a_1}{m(m+1)} \geq 0
\end{equation}
we show that 
\begin{equation}
     {IRV}^{\boldsymbol{E}}_{(2)}(v_1) - {IRV}^{\boldsymbol{E}}_{(1)}(v_1) \leq \left(\beta - {App}^{\boldsymbol{E}}_{v_1}(c_1)\right)\left(\frac{m-a_1}{m(m+1)}\right) \leq 0
\end{equation}\\

Therefore, we have shown that case (2), where the voter does not vote for their highest weighted candidate, will always result in that voter's Individual Reward Value being lower than or equal to that of case (1), where the voter votes for their highest weighted candidate


\section{Voting Strategies and their Consequences}
\subsection{Grouping the voters}
An important observation is that we can simplify our task of analysing voting strategy performance by grouping voters in accordance with their voting behaviors under the strategy. For example, for the Voting Strategy where voters vote for only the candidate which they've assigned the highest approval weight out of all the candidates, we can group voters into groups according to which candidate they will be voting for. We can then say that there are a certain number of voters in the group, with each voter having an approval weight for a given candidate that is the averaged weight of that candidate among all of the input voters that gave rise to the group.

We can show how such grouping can be used while maintaining the integrity of the PRV value.

Let there be a set of candidates $\boldsymbol{C}$ and voters $\boldsymbol{V}$.

\begin{equation*}
    PRV = \sum^{}_{v \in \boldsymbol{V}}{\left(\sum^{}_{c \in \boldsymbol{C}}{{App}_v(c) \cdot {Res}(c)}\right)}
\end{equation*}

Let us group voters together based on a certain criterion, we will be using their highest ranked candidate for this example.
Let a set of voters that have $c$ as their highet ranked candidate be $V_c$.

\begin{equation*}
\begin{aligned}
PRV &= \sum^{}_{c \in \boldsymbol{C}}{\left(\sum^{}_{v \in V_c}{{IRV}(v)}\right)}\\
&= \sum^{}_{c \in \boldsymbol{C}}{\left({IRV}({v_c}_1)+{IRV}({v_c}_2)+\cdots\right)}\\
&= \sum^{}_{c \in \boldsymbol{C}}{\left(\sum^{}_{c in \boldsymbol{C}}{\left({App}_{{v_c}_1}(c)\cdot Res(c)\right)}+\sum^{}_{c in \boldsymbol{C}}{\left({App}_{{v_c}_2}(c)\cdot Res(c)\right)}+\cdots\right)}\\
&= \sum^{}_{c \in \boldsymbol{C}}{\left(\sum^{}_{c in \boldsymbol{C}}{\left(\sum^{}_{v \in V_c}{{App}_{v}(c)}\right)}+\right)}
\end{aligned}    
\end{equation*}


\subsection{Voting only for the highest weighted candidate}
We will now show how and why we can use such an approach to model elections where only the Highest Approval voting strategy is used, whilst still preserving the PRV value.\\

Let there be an election with $|c|$ candidates. We can construct $|c|$ groups where $|g_c|$ represents a group that supports (will vote for) candidate $c \in \boldsymbol{C}$. We will give each group a population size $|g_c|$ that represents the number of voters that will vote for candidate $c$. The group will be allocated approval weights $\forall c \in \boldsymbol{C}$ that corresponds to the average Approval Weight for each given candidate among the voters associated with that group. Let the set of all such groups be $\boldsymbol{G} = \{g_c | c \in \boldsymbol{C}\}$.

For our theoretical cases, without loss of generality, we can notate the Social Optimal candidate, ${argmax}_{c \in \boldsymbol{C}} \sum^{}_{v \in \boldsymbol{V}}{App^{\boldsymbol{E}}_{v}(c)}$, as $c_1$.

\begin{equation}
     PRV_{SO} = \sum^{}_{g \in \boldsymbol{G}}{\left(App_g(c_1)\cdot|g|\right)} 
\end{equation}

In the strategy where each voter votes only for their top weighted candidate, the result of the election will be proportional to the population size of each group.

\begin{equation}
     PRV_{FirstOnly} = \sum^{}_{g \in \boldsymbol{G}}{\left(\sum^{}_{c \in \boldsymbol{C}}{\left(\frac{|g_c|}{|\boldsymbol{V}|} \cdot App_g(c)\right)} \cdot |g|\right)} 
\end{equation} 

We can create a metric called "Difference" (or "diff" for short) that is the difference between $PRV_{SO}$ and $PRV_{FirstOnly}$.
\begin{equation}
\begin{aligned}
{diff} &= PRV_{FirstOnly} - PRV_{SO}\\
&= \sum^{}_{g \in \boldsymbol{G}}{\left(\sum^{}_{c \in \boldsymbol{C}}{\left(\frac{|g_c|}{|\boldsymbol{V}|} \cdot App_g(c)\right)} \cdot |g| - \sum^{}_{g \in \boldsymbol{G}}{\left(App_g(c_1)\cdot|g|\right)}\right)}\\
&= \sum^{}_{g \in \boldsymbol{G}}{\left(\sum^{}_{c \in \boldsymbol{C}}{\left(\frac{|g_c|}{|\boldsymbol{V}|} \cdot App_g(c)\right)} \cdot |g| - App_g(c_1)\cdot|g|\right)}\\
&= \sum^{}_{g \in \boldsymbol{G}}{\left(|g|\left(\sum^{}_{c \in \boldsymbol{C}}{\left(\frac{|g_c|}{|\boldsymbol{V}|} \cdot App_g(c)\right)} - App_g(c_1)\right)\right)}
\end{aligned}
\end{equation}\\

Our goal will be to minimise this $diff$ value.




















\end{document}