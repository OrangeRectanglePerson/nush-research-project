\documentclass{article}
\usepackage[utf8]{inputenc}
\usepackage{amsmath}
\usepackage{graphicx} 
\graphicspath{ {./Images/} } %tells the program what folder to find the images in

\title{Proportional Approval Voting\\DRAFT}
\author{
  Lim Devon\\
  \texttt{h2010076@nushigh.edu.sg}
  \and
  Lee Hong Hao Casper\\
  \texttt{h2010069@nushigh.edu.sg}
  \and
  Kang Jun Jie Joseph\\
  \texttt{h2010059@nushigh.edu.sg}
}
\date{\today}

\begin{document}

\maketitle

\section{Introduction}
The purpose of this document is to test out some of the features of \LaTeX{}. Compile this document, read the notes and compare the code to the final product to learn how to use these commands.

\section{Equations}
This section will be testing some different ways to write equations in different ways. 

\subsection{Simple Equations}
Here is an example of an equation
%using the command \begin{equation} always numbers the equation, label just allows you to reference later%
\begin{equation} \label{luna}
P(x)=ax^2+bx+c
\end{equation}

The equation \ref{luna} is an example of a quadratic equation.

\subsection{Splitting an equation}

%This is how you can add a comment without it showing in your final document.  The \\ starts a new line. The & symbol tells the program where you want your two equations to align. split allows you to label the group of equations%
\begin{equation}\label{eq2}
\begin{split}
V&=\frac{4}{3}\pi r^3 \\
&= \frac{4\pi r^3}{3}
\end{split}
\end{equation}

Equation \ref{eq2} shows that you can write the equation for the volume of a sphere in multiple ways.

\subsection{Aligning}

Here is an example of aligning the equations with the = signs.
\begin{align*} 
2x+3&=3\\
3x+2&=2x+3
\end{align*}\\

Below is another example but with multiple equations.
\begin{align*}
2x&=3y  &   A&=B    &   3g&=h+4 \\
x+2&=y  &   A&=2B+1 &   g+1&=2h
\end{align*}\\

Here is the last example of this section where the equations are centered but not aligned.
\begin{gather*} 
5x+3=4y-3 \\
2x=y+1 
\end{gather*}\\


\subsection{Other ways of writing math}
%There are some more basic commands to write math as well such as \( \), \[ \], $ $, or \begin{math} \end{math}
You can also write equations such as \(A=\pi r^2\) in line like this or an equation such as $A=l*w$ in line using the other command that you can see in the code.\\\\

To write equations not in line you can also write it like this: \[A=\frac{1}{2}h(b_1+b_2)\]\\\\\\

\section{Questions}

Let there be an election \pmb{E} = (\pmb{C}, \pmb{V}, \pmb{A}).
\pmb{C} is the set of all candidates.
\pmb{V} is the set of all voters.
\pmb{A} is the set of all of the voters' Approval Value Functions.\\

$App^{E}_{v}(c)$ is an Approval Value Function for voter $v \in \pmb{V}$ in election $\pmb{E}$.

\begin{equation*}
App^{E}_{v} : \pmb{C} \mapsto [-1,1]
\end{equation*}\\

Where a negative value means the voter believes the candidate will have a bad impact, a positive value means the voter believes the candidate will have a positive impact, and 0 means the voter is apathetic towards the candidate. -1 is the worst possible impact and +1 is the best possible impact a candidate may have. It is not neccessary that the least liked and most liked candidates are placed at -1 and +1 respectively, they may just be assigned an arbitrary value between -1 and 1.\\

$Res^{E}_{S}(c)$ is an function, that given election \pmb{E} and voting strategy \pmb{S}, provides the result of the vote for candidate $c \in \pmb{C}$.

\begin{gather}
Res^{E}_{S} : \pmb{C} \mapsto [0,1]\\
\sum^{}_{c \in \pmb{C}}{Res^{E}_{S}(c)} = 1
\end{gather}\\

\subsection{Question 1}

Let there be a Socially Optimal outcome. It will be defined as the case where
\begin{gather}
App^{E}_{pop}(c) = \sum^{}_{v \in \pmb{V}}{App^{E}_{v}(c)}\\
c_{w} = {argmax}_{c \in \pmb{C}} App^{E}_{pop}(c)\\
Res^{E}_{S}(c_{w}) = 1
\end{gather}\\

Let there be a Social-Positive Optimal outcome. It will be defined as the case where the value of the function below is maximised.
\begin{equation}
\sum^{}_{v \in \pmb{V}}{\left(
\begin{cases}
1, & \sum^{}_{c \in \pmb{C}}{(Res^{E}_{S}(c) \cdot App^{E}_{v}(c))} > 0\\
0, & {otherwise}
\end{cases}
\right)}
\end{equation}\\

Let us define the Population Reward Value as
\begin{equation}
\sum^{}_{v \in \pmb{V}}{\left(
\sum^{}_{c \in \pmb{C}}{\left(
Res^{E}_{S}(c) \cdot App^{E}_{v}(c)
\right)}
\right)}
\end{equation}\\

Are there bounds (either uppor or lower) on the difference between the population reward values of the Soically Optimal and Social-Positive Optimal outcomes?


\subsection{Some other stratgies and outcomes}

\subsubsection{Outcomes}

Let there be a Social-Sign Optimal outcome. It will be defined as the case where the value of the function below is maximised.
\begin{equation}
\sum^{}_{v \in \pmb{V}}{\left(
\begin{cases}
1, & \sum^{}_{c \in \pmb{C}}{(Res^{E}_{S}(c) \cdot App^{E}_{v}(c))} > 0\\
-1, & \sum^{}_{c \in \pmb{C}}{(Res^{E}_{S}(c) \cdot App^{E}_{v}(c))} < 0\\
0, & {otherwise}
\end{cases}
\right)}
\end{equation}\\

Let there be a Social-Negative Optimal outcome. It will be defined as the case where the value of the function below is maximised.
\begin{equation}
\sum^{}_{v \in \pmb{V}}{\left(
\begin{cases}
-1, & \sum^{}_{c \in \pmb{C}}{(Res^{E}_{S}(c) \cdot App^{E}_{v}(c))} < 0\\
0, & {otherwise}
\end{cases}
\right)}
\end{equation}\\

\subsubsection{Strategies}

Let there be a voting strategy where a voter only votes for their highest ranked options.

Let there be a voting strategy where a voter only votes for all candidates except their lowest ranked option.

Let there be a voting strategy where a voter only votes for the options they ranked positively.

\subsubsection{More Questions!}

Are there bounds between the social cost of these strategies and outcomes compared to the utilitarian Socially Optimal outcome?\\
Are there bounds between the difference of social costs among these outcomes?



\end{document}